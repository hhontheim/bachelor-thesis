\chapter{Einleitung}

Zwischen 2015 und 2050 wird die Zahl der Menschen, die älter als 60 Jahre sind, von 900 Millionen auf 2 Milliarden steigen. Dies entspricht einem Anstieg von $ 12 \% $ auf $ 22 \% $ der gesamten Weltbevölkerung\footnote{\url{https://www.who.int/features/factfiles/ageing/en/}}.

Werden ältere Personen mit neuen Technologien, wie etwa Smartphones, konfrontiert, herrscht oft eine gewisse Abneigung. Der häufigste Grund sind Bedenken hinsichtlich der Privatsphäre und des Datenschutzes. Ältere möchten oft nicht viel von sich preisgeben und haben Angst, dass diese Daten an die Öffentlichkeit gelangen könnten. Oft sind es aber auch mangelnde Kenntnisse und unzureichende Ausrüstung, die den Zugang zu diesen Technologien erschweren oder die Tatsache, dass Anwendungen selten an die Bedürfnisse und Interessen dieser angepasst sind \cite{Almeida:2015:Recommendations-for-the-Development-of-Web-Interfaces}\cite{Boll:2015:User-Interfaces-with}\cite{Leonardi:2010:An-Exploratory-Study-of-a-Touch-Based}.

Im Rahmen des Projekts \doublequotes{UrbanLife+}\footnote{\url{https://www.urbanlifeplus.de}} der Universität der Bundeswehr München, wird versucht die Teilhabe von Senioren im öffentlichen Raum zu verbessern. Mit dieser Arbeit möchten wir untersuchen, welche Richtlinien es bei Benutzerschnittstellen für ältere Personen gibt, indem wir bereits existierende Arbeiten analysieren und daraus Best Practices ableiten.

Im Anschlus soll eine mobile Anwendung für das iPhone und die Apple Watch entwickelt werden, die eine Identifikation im urbanen Raum über \emph{Bluetooth Low Energy} ermöglicht.

Allein aus Gr"unden der Lesbarkeit wird auf die gleichzeitige Verwendung mehrerer geschlechtsspezifischer Sprachformen verzichtet. S"amtliche Personenbezeichnungen gelten f"ur alle Geschlechter.